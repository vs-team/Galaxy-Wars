\documentclass[10pt,a4paper]{article}
\usepackage[utf8]{inputenc}
\usepackage{amsmath}
\usepackage{amsfonts}
\usepackage{amssymb}
\usepackage{graphicx}
\author{Justin van den Anker, Wouter Volders, Hidde Bolijn, Martijn de Lange}
\title{Cooking Game}
\begin{document}
\maketitle

\section{Enviroment}
The concept of this game has been developed for the DAF augmented reality lab of Tilburg University. This concept is to show off the capabilities of the augmented reality lab and to teach people to work together in order to win the game.

\section{Setting}
The setting of this game is based on stress. The concept is a diner as seen from the employees eyes. Customers will come to the diner, ordering food. The objective for the players is to deliver the order. This means they have to prepare the food, starting at the order of the customer, to the serving of the plates. They are in charge of everything behind the scenes. The players will have to work together to make the diner a success. Through good teamwork and communication the diner can be managed. The diner starts out slowly but as you server customers, the difficulty will slowly increase, increasing the pressure on the players. More people will come and the orders become more complex. The better the team is at communication, working together and handling stress, the better they will be at this game.

\section{Gameplay}
The gameplay of this game is somewhat unique, since it's a real life kitchen experience. On each of the walls there will be a projection of different areas of the kitchen. The first wall will project the counter. The counter is where the players will have to take orders from the customers and has to make sure that he gets the orders right. The second wall will project the preparation area. Here the players will have to make sure that the right ingredients are ready to be used in the cooking area. The third wall will contain the cooking area of the kitchen. In this area there will be multiple dishes cooking on the stove and the players have to make sure the dishes are underdone or burnt. Once the dishes have been cooked just right, they have to be moved to the garnish area. The fourth wall is where the garnish area is located. This area allows the players to present the food on a plate and making sure it finds its way to the right customer. After the customers finish their meal, they will leave feedback for the restaurant. Based on the players' performance, they will receive points. At the end, the collective score of the team will be shown on the screen and moved into the ranking.

\section{Implementation}
This solution will be implemented in Unity with Casanova if possible. The exact controllers that will be used are as of yet unknown since it is not known if the controllers from the augmented reality lab support Unity. There is however an alternative that does have Unity support, this controller is called the Razer Hydra and supports 6 degrees of movement. The rest of the implementation is unknown as of yet until we can speak to the technical staff of the lab about the possibilites.

\section{Issues}
\begin{itemize}
\item How would Unity be used in the augmented reality lab?
\item What can the controllers do and do they have Unity support?
\item What is the latency between the slave computers and master computer?
\end{itemize}

\end{document}