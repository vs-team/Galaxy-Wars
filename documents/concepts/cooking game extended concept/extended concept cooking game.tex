\documentclass[10pt,a4paper]{article}
\usepackage[utf8]{inputenc}
\usepackage{amsmath}
\usepackage{amsfonts}
\usepackage{amssymb}
\usepackage{graphicx}
\author{Justin van den Anker, Wouter Volders, Hidde Bolijn, Martijn de Lange}
\title{Cooking Game}
\begin{document}
\maketitle

\section{Enviroment}
The concept of this game has been developed for the DAF augmented reality lab of Tilburg University. This concept is aimed to show off the capabilities of the augmented reality lab and to teach people to work together in order to win the game. 

\section{Setting}
The setting of this game is based on stress. The concept is a diner as seen from the employees eyes. The players will serve the people in the restaurant. This means they have to prepare the food from order to serving the plates, they are in charge of everything behind the scenes. The players will have to work together to make the diner a succes. Through good teamwork and communication the diner can be managed. Instead of going out to eat, the players will serve the people that come for dinner. This concept can be extremely stressfull since the pressure on the players can be increased. The diner starts out slowly but as you gain experience and get better at the job, the popularity of the diner grows and gets more attention. The better the team is at handling stress, the better they will be at this game.

\section{Gameplay}
The gameplay of this game is somewhat unige, since there will be a real like kitchen experience. On each of the walls there will be a projection of different areas of the kitchen. The first wall will project the counter, the player that is responsible for this wall, will have to take the orders from the people and has to make sure that he gets the orders right. The second wall will project the preperation are, the player responsible for this wall will have to make sure that al the right ingredients are ready to be used by the chef. The third wall will show the cooking area of the chef. In this area there will be multiple dishes cooking on the stove and the chef has to make sure this dishes do not over nor undercook. Once the dishes have been cooked just right, they will be moved on to the server, this is the person that is in charge of nicely presenting the food on a plate and making sure it finds its way to the right table. After the customers finish there meal, they will leave feedback for the restaurant. Extra points can be earned by excelling in one part of the kitchen and if all parts of the dish were handled perfectly the team will receive a bonus. Once the time runs out, the collective score of the team will be shown on the screen and moved into the ranking.

\section{Implementation}
This solution will be implemented in Unity with Casanova if possible. The exact controllers that will be used are as of yet unknown since it is not known if the controllers from the augmented reality lab support Unity. There is however an alternative that does have Unity support, this controller is called the Razer Hydra and supports 6 degrees of movement. The rest of the implementation is unknown as of yet until we can speak to the technical staff of the lab about the possibilites.

\section{Issues}
\begin{itemize}
\item How would Unity be used in the augmented reality lab?
\item What can the controllers do and do they have Unity support?
\item What is the latency between the slave computers and master computer?
\end{itemize}

\end{document}