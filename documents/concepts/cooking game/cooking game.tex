\documentclass[10pt,a4paper]{article}
\usepackage[utf8]{inputenc}
\usepackage{amsmath}
\usepackage{amsfonts}
\usepackage{amssymb}
\usepackage{graphicx}
\author{Justin van den Anker, Wouter Volders, Hidde Bolijn, Martijn de Lange}
\title{Concept Cooking Game}
\begin{document}
\section{Setting}
The purpose of this game is to get people to work together based on continuity, if any player at any given point does not do his part it will result in failing the game. The game is a simulation of a restaurant, each player will handle one part of the restaurant. The first player is responsible for handling the orders and payment, this means he has to take the orders and give them to the chef. This brings us to the next player, the chef is in charge of cooking the food that is requested by the customers, he has to put the right ingredients in the right dishes. Then there is one player left, this player is responsible for preparing the dishes to be served out. The food that the chef has prepared needs to be put on plates in a decent manner, the better the dish is presented, the more points will be received. The feedback from the customer about the dish will decide the amount of points the team receives. If all players participate properly in the game bonuses can be added for overall performance. The game will have to be played using the controllers that were delivered with the augmented reality lab. 

\section{Issues}
One of the issues that could present itself is the screen tearing on the edges of the wall, this can be solved by making part of the kitchen a static enviroment. In this manner the edges of the screen do not need changing and there will be no screen tearing.
\end{document}