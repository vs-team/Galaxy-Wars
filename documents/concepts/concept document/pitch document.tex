\documentclass[10pt,a4paper]{article}
\usepackage[utf8]{inputenc}
\usepackage{amsmath}
\usepackage{amsfonts}
\usepackage{amssymb}
\usepackage{graphicx}
\author{Justin van den Anker, Wouter Volders, Hidde Bolijn, Martijn de Lange}
\title{Concepts Augmented Reality Lab}
\begin{document}
\maketitle
\section{Assignment}
At tilburg University DAF has just opened an augmented reality lab, this lab however has no showcase games as of yet. Four students from the Rotterdam University of Applied Sciences are assigned the task of creating a game for this lab. 

The requirements for the game are the following:
\begin{itemize}
\item Show off the capabilities of the augmented reality lab
\item Create a game that forces people to work together in the augmented reality lab.
\end{itemize}

\section{Concept 1: Survival Game}
The general idea behind this idea is to make people work together in a stressfull enviroment. The game would take place from the inside of a vehicle. The player will reside in this vehicle and are expected to fend off oncoming attackers. Each player has it's own responsibilities in the truck. There are three unique rolls for players. First there is the driver, this person is responsible for moving the vehicle through the field of oncoming attackers and making sure to keep the damage done to the vehicle to a minimum. The second roll is that of the spotter, this person is in charge of a flashlight to spot oncoming attackers coming from the sides and back of the vehicle and inform the rest of the team of the location of such attackers. The last roll in the vehicle as that of the gunner. This person is responsible for shooting the attackers and making sure that the oncoming attackers are shot down and do not damage the vehicle. The game is over when the vehicle is damaged too much and the attackers can get to you. 

\section{Concept 2: Cooking Game}

The purpose of this game is to get people to work together based on continuity, if any player at any given point does not do his part it will result in failing the game. The game is a simulation of a restaurant, each player will handle one part of the restaurant. The first player is responsible for handling the orders and payment, this means he has to take the orders and give them to the chef. This brings us to the next player, the chef is in charge of cooking the food that is requested by the customers, he has to put the right ingredients in the right dishes. Then there is one player left, this player is responsible for preparing the dishes to be served out. The food that the chef has prepared needs to be put on plates in a decent manner, the better the dish is presented, the more points will be received. The feedback from the customer about the dish will decide the amount of points the team receives. If all players participate properly in the game bonuses can be added for overall performance. The game will have to be played using the controllers that were delivered with the augmented reality lab. 

\section{Concept 3: Puzzle Game}
For the puzzle games the players would have to work together using their minds. The puzzles will require all the players to participate in the thinking part. One of the walls in the room will show a question that the players need to answer. If the participants cannot answer this question they will need to use the controller as a candle and walk past the other three dark walls. When the controller is near the wall, the wall will partially light up. In this manner the players can search for clues to the answer of the question. Searching for the clues however, costs time and depletes the amount of points the players will receive for answering the question.

\section{Issues}
There might be issues encountered when it comes to the synchronizing of the frame rate of the multiple projectors. This problem might possibly be solved by using the beams of the vehicle as edges of the screen to prevent screen tearing. Another issue might be using the controller for the screens in Unity. There might be the need to write scripts so that Unity understands the input of the controls. The last issue that lies ahead is the synchronizing of the game logic, this has to happen in such a manner that every screen is updated real time so that there is no delay when the drivers changes the direction of the vehicle. The screens have to move all at the same time, otherwise there would be a gicantic distortion in the movement of the screens. This problem might have to be solved by using the master computer as a director for the game logic.

\end{document}