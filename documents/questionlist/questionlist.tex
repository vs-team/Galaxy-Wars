\documentclass[10pt,a4paper]{article}
\usepackage[utf8]{inputenc}
\usepackage{amsmath}
\usepackage{amsfonts}
\usepackage{amssymb}
\usepackage{graphicx}
\author{Justin van den Anker, Wouter Volders, Hidde Bolijn, Martijn de Lange}
\title{Questions}
\begin{document}
\maketitle

\begin{itemize}
\item Is Unity compatible with the augmented reality lab and if so how integrated is it?
\item If Unity is compatible, how would we go about the setup for the scene? Which computer needs to actually run the scene?
\item Would it be possible to run the scene on the master computer and then use different Unity Camera's for the projection on the four screens?
\item Are the controllers compatible with Unity and what are the capabalities of said controllers?
\item In regard to the input of the controllers, is there latency involved? Also, how would the input best be handled, through the slave or master PC?
\item Can we synchronize the frame rate between the screens while using Unity?
\item What would be a good setup for the multiplayer enviroment on the computers in respect to the slave computers and the master computer?
\item Is it possible to place extra sensors from outside companies for more controllers?
\item Would it be possible to track players in the room quite accurately.

\end{itemize}
The answers of these questions will greatly impact the concept that will eventually be turned into an actual game. Depending on the answers a choice will have to be made to go with the survival game or the cooking game.

Both games have their advantages when it comes to the augmented reality lab:

Survival Game:
\begin{itemize}
\item Easier to build in respect to the amount of aspects the game needs to have, it is in a sense a survival game with simple and clear gameplay.
\item There is less complicated art needed to fill the scene.
\end{itemize}

Cooking Game:
\begin{itemize}
\item The scene is static and thus requires no synchronizing of the edges to prevent tearing. The edges will stay smooth and in one smooth color. This could be very positive if it is not possible to synchronize the framerate in Unity.
\item The gameplay is interesting for all ages.
\end{itemize}

Based on the information written above, what would be the best course of action with the biggest chance of success?
\end{document}